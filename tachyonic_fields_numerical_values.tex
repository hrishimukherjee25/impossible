
\documentclass{article}
\usepackage{amsmath}
\usepackage{amssymb}

\title{Numerical Values of Tachyonic Fields}
\author{}
\date{}

\begin{document}

\maketitle

\section*{Introduction}

This document presents the numerical values for the tachyonic fields across different layers of the warp drive propeller at specific positions. These values were calculated based on the positions along the propeller and the corresponding field oscillations.

\section*{Layer 1}
\begin{align*}
T_{\text{Layer 1}}(x = 0) & = 0, \\
T_{\text{Layer 1}}(x = 111.11) & = -0.999996, \\
T_{\text{Layer 1}}(x = 222.22) & = 0.008726, \\
T_{\text{Layer 1}}(x = 333.33) & = 0.999980, \\
T_{\text{Layer 1}}(x = 444.44) & = -0.017452, \\
T_{\text{Layer 1}}(x = 555.56) & = -0.999962, \\
T_{\text{Layer 1}}(x = 666.67) & = 0.026178, \\
T_{\text{Layer 1}}(x = 777.78) & = 0.999939, \\
T_{\text{Layer 1}}(x = 888.89) & = -0.034904, \\
T_{\text{Layer 1}}(x = 1000) & = -0.999912.
\end{align*}

\section*{Layer 2}
\begin{align*}
T_{\text{Layer 2}}(x = 0) & = 0, \\
T_{\text{Layer 2}}(x = 111.11) & = 0.999924, \\
T_{\text{Layer 2}}(x = 222.22) & = -0.017452, \\
T_{\text{Layer 2}}(x = 333.33) & = -0.999696, \\
T_{\text{Layer 2}}(x = 444.44) & = 0.034904, \\
T_{\text{Layer 2}}(x = 555.56) & = 0.999319, \\
T_{\text{Layer 2}}(x = 666.67) & = -0.052336, \\
T_{\text{Layer 2}}(x = 777.78) & = -0.998792, \\
T_{\text{Layer 2}}(x = 888.89) & = 0.069756, \\
T_{\text{Layer 2}}(x = 1000) & = 0.998115.
\end{align*}

\section*{Layer 3}
\begin{align*}
T_{\text{Layer 3}}(x = 0) & = 0, \\
T_{\text{Layer 3}}(x = 111.11) & = 0.999912, \\
T_{\text{Layer 3}}(x = 222.22) & = -0.034904, \\
T_{\text{Layer 3}}(x = 333.33) & = -0.999649, \\
T_{\text{Layer 3}}(x = 444.44) & = 0.069756, \\
T_{\text{Layer 3}}(x = 555.56) & = 0.999230, \\
T_{\text{Layer 3}}(x = 666.67) & = -0.104528, \\
T_{\text{Layer 3}}(x = 777.78) & = -0.998651, \\
T_{\text{Layer 3}}(x = 888.89) & = 0.139173, \\
T_{\text{Layer 3}}(x = 1000) & = 0.997913.
\end{align*}

\end{document}
